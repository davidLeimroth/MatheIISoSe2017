%
% Mitschrieb der Mathe II Vorlesung - Albert-Ludwigs-Universität Freiburg, Technische Fakultät, Informatik
%
% Dieses Dokument erhebt keinen Anspruch auf Vollständigkeit und/oder Korrektheit
%
% @Author: David Leimroth
% @Date:   2017-04-30 17:15:31
% @Last Modified by:   David Leimroth
% @Last Modified time: 2017-05-01 00:45:57
%
% Mitschrieb der Mathe II Vorlesung - Albert-Ludwigs-Universität Freiburg, Technische Fakultät, Informatik
%
% Dieses Dokument erhebt keinen Anspruch auf Vollständigkeit und/oder Korrektheit
%
% @Author: David Leimroth
% @Date:   2017-04-30 17:15:31
% @Last Modified by:   David Leimroth
% @Last Modified time: 2017-04-30 23:40:42
%
% Mitschrieb der Mathe II Vorlesung - Albert-Ludwigs-Universität Freiburg, Technische Fakultät, Informatik
%
% Dieses Dokument erhebt keinen Anspruch auf Vollständigkeit und/oder Korrektheit
%
% @Author: David Leimroth
% @Date:   2017-04-30 17:15:31
% @Last Modified by:   David Leimroth
% @Last Modified time: 2017-04-30 23:40:42

 %----------------------------------------------------------------------------------------
% PACKAGES UND ANDERE DOKUMENT KONFIGURATIONEN
%----------------------------------------------------------------------------------------
\documentclass[bibtotoc,11pt,a4paper,]{scrartcl}
\usepackage[utf8x]{inputenc}
\usepackage[ngerman]{babel}
\usepackage{graphicx}%Bilder einfügen
\usepackage{tocstyle} %Inhaltsverzeichnis gepunktet
\usepackage{setspace} %Zeilenabstand 1,5
    \setstretch{1.28}%  für 12pt
\usepackage{geometry} %Seitenränder
  \geometry{a4paper, top=20mm, left=25mm, right=25mm, bottom=25mm}
  \parindent0cm %Blocksatz nicht eingerückt
\usepackage{paralist}
\usepackage{microtype} %besser Blocksatz
\usepackage[hang, labelfont=bf, textfont={small}, width=12.5cm, up ]{caption} %Tabelle-/Abbildungsunterschriften Konfiguration
%\usepackage{booktabs} %nur horizontale Linien in Tabellen
\usepackage{amsfonts}%Mathe
\usepackage{amsmath}%Mathe
\usepackage{pxfonts}
\usepackage{upgreek}
\usepackage{float}
\usepackage{caption}
\usepackage{subcaption}






%----------------------------------------------------------------------------------------
% EIGENE BEFEHLE
%----------------------------------------------------------------------------------------
\newenvironment{listeP}{\begin{itemize} \itemsep -26pt}{\end{itemize}} %Liste mit Punkten
\newenvironment{listeN}{\begin{enumerate} \itemsept -26pt}{\end{enumerate}} %Liste mit Zahlen



\bibliographystyle{unsrt} %Literaturverzeichnis Stil

%----------------------------------------------------------------------------------------
% KOPF- UND FUSSZEILE
%----------------------------------------------------------------------------------------

\usepackage{fancyhdr} % Headers and footers
\renewcommand{\sectionmark}[1]{\markright{#1}}
\pagestyle{fancy}
\cfoot{\thepage}
\renewcommand{\headrulewidth}{0.5pt}
\renewcommand{\sectionmark}[1]{\markright{#1}}
\renewcommand{\subsectionmark}[1]{}
\renewcommand{\subsubsectionmark}[1]{}
\lhead{\scriptsize{\textsc{ANPASSEN $\bullet$ anpassen}}}
\rhead{\textbf{\textsc{\rightmark}}}


\setcounter{page}{0}


%----------------------------------------------------------------------------------------
% TITELSEITE KONFIGURATION
%----------------------------------------------------------------------------------------

\titlehead{\includegraphics[scale=0.46]{}} %Bilder/LOGO.JPG
\title{\textbf{title}}
\subject{Bericht - Veranstaltung}






%\documentclass{article}
\begin{document}

  \section{Grundlegende Algebraische Strukturen}
    \subsection{Struktur}
      Nicht leere Menge M mit Operationen/Verknüpfungen/Funtionen. z.B. eine zweistellige Operation:
      \begin{equation}
        o: M \times M \rightarrow M
      \end{equation}
      oder einstellige Operationen $M \rightarrow M$ (z.B. $^z \rightarrow -z\ in\ \mathbb{Q}$)
      \\
      vierstellig: $M\ =\ \mathbb{R}$
      \begin{equation}
        (r_1, r_2, r_3, r_4) \rightarrow (r_1 + r_2) \times (r_3 + r_4)
      \end{equation}
      nullstellig:
      \begin{equation}
        M^0 \rightarrow M\ \ \ \ |M^0| = 1
      \end{equation}

      Die nullstellige Operation wird identifiziert mit dem Bild $\in$ M, d.h. mit einer Konstanten aus M.\\

      Sei M eine endliche Menge\\
      $|M^n|$\\
      
      \begin{equation}
        M^2 = M \times M = \{(m_1, m_2)\ |\ m_i \in M\}\\
        \forall \ n \le 1 : |M^n| = |M|^n  
      \end{equation}
      
      Also möchte man $|M^0| = |M|^0 = 1$

      Operationen $M^n \rightarrow M$ heißen ``innere Operationen''
      
      Beispiel einer ``äußeren Operation'':\\
      V Vektorraum über den reellen Zahlen\\
      Skalarmultiplikaton
      Vektor $v \in V$\\
      Skalar $r \in \mathbb{R}$\\
      $r \times v \in V$\\
      
      Bsp.: $(\mathbb{N}, +),\ (\mathbb{N}, \times),\ (\mathbb{N}, +, \times),\ (\mathbb{Z}, +),\ (\mathbb{Z}, +, \circ)$
      
      Sei M eine Menge\\
      Abb(M, M) = Menge der Abbildungen (Funktionen)\\
      (Abb(M, M), $\circ$) $\circ$ ist eine Verknüpfung (Hintereinanderausführung) von Abbildungen\\
      \begin{equation}
          (f \circ g)(m) = f(g(m))
      \end{equation}
      
      Sei A eine Menge von Symbolen (``Alphabet'')\\
      Sei $A^*$ die Menge der ``Wörter'' über A, d.h. endliche Folgen von Symbolen aus A\\
      
      Bsp.: A = \{0, a, b, \#\}\\
      aa\#0, b0b, \#\#\#, a $\in a^*$
      
      $\lambda, \epsilon$ ``leere Wort''
      
      $(a^*, ^{\smallfrown})$ $^{\smallfrown}$ ist die Konkatenation von Wörtern (Hintereinanderschreibung)
      
      aa0\#$^{\smallfrown}$b0b = aa0\#b0b
      
      \subsubsection{Eigenschaften von zweistelligen Operationen}
        
        Sei M eine Menge\\
        $\circ, \ast:\ M^2 \rightarrow M$
        
        \begin{enumerate}
            \item $\circ$ heißt kommutatic, falls $\forall m_1, m_2 \in M$ gilt:
            \begin{equation*}
                m_1 \circ m_2 = m_2 \circ m_1
            \end{equation*}
            
            \item $\circ$ heißt assoziativ, falls $\forall m_1, m_2, m_3 \in M$ gilt:
            \begin{equation*}
                (m_1 \circ m_2) \circ m_3 = m_1 \circ (m_2 \circ m_3)
            \end{equation*}
            
            \item $\star$ heißt distributiv über $\circ$, falls $\forall m_1, m_2, m_3 \in M$ gilt:
            \begin{itemize}
                \item[a)] $(m_1 \circ m_2) \star m_3 = (m_1 \star m_3) \circ (m_2 \star m_3)\ \leftarrow$ rechtsdistributiv
                \item[b)] $m_3 \star (m_1 \circ m_2) = (m_3 \star m_2) \circ (m_3 \star m_2)\ \leftarrow$ linksdistributiv
            \end{itemize}
            
            \item $e \in M$ heißt neutrales Element von $\circ$, falls $\forall m \in M$ gilt:
            \begin{itemize}
                \item[a)] $e \circ m = m\ \leftarrow$ linksneutral
                \item[b)] $m \circ e = m\ \leftarrow$ rechtsneutral
            \end{itemize}
            
            \item Fall $e \in M$ neutrales Element von $\circ$ ist, dann ist m' inverses Element von m bzgl. $\circ$ (und e), falls gilt:
            \begin{itemize}
                \item[a)] $m \circ m' = e$
                \item[b)] $m' \circ m = e$
            \end{itemize}
            
            0 ist das neutrale Element der Addition in $\mathbb{N}/\mathbb{Z}$\\
            1 ist das neutrale Element der Multiplikation in $\mathbb{N}/\mathbb{Z}$
        \end{enumerate}
      
      
      Definition: Ein Monoid besteht aus einer nicht-leeren Menge M mit einer assoziativen Verknüpfung $\circ$ mit neutralem Element e.\\
      $(M, \circ, e)$\\
      $(M, \circ)$\\
      M heißt kommutatives Monoid, falls $\circ$ kommutativ ist.
      
      Bsp.:

\end{document}