%
% Mitschrieb der Mathe II Vorlesung - Albert-Ludwigs-Universität Freiburg, Technische Fakultät, Informatik
%
% Dieses Dokument erhebt keinen Anspruch auf Vollständigkeit und/oder Korrektheit
%
% @Author: David Leimroth
% @Date:   2017-04-30 17:15:31
% @Last Modified by:   David Leimroth
% @Last Modified time: 2017-04-30 18:24:46

\documentclass{article}
\begin{document}

  \section{Grundlegende Algebraische Strukturen}
    \subsection{Struktur}
      Nicht leere Menge M mit Operationen/Verknüpfungen/Funtionen. z.B. eine zweistellige Operation:
      \begin{equation}
        o: M \times M \rightarrow M
      \end{equation}
      oder einstellige Operationen $M \rightarrow M$ (z.B. $^z \rightarrow -z\ in\ \mathbb{Q}$)
      \\
      vierstellig: $M\ =\ \mathbb{R}$
      \begin{equation}
        (r_1, r_2, r_3, r_4) \rightarrow (r_1 + r_2) \times (r_3 + r_4)
      \end{equation}
      nullstellig:
      \begin{equation}
        M^0 \rightarrow M\ \ \ \ |M^0| = 1
      \end{equation}

      Die nullstellige Operation wird identifiziert mit dem Bild $\in$ M, d.h. mit einer Konstanten aus M.

      M endliche Menge\\
      $|M^n|$\\
      $M^2 = M \times M = \{(m_1, m_2) | m_i \in M\}$ für n \le 1 : $\|M^n^\| = \|M\|^n$\\


\end{document}